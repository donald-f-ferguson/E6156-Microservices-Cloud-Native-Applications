
% Default to the notebook output style

    


% Inherit from the specified cell style.




    
\documentclass[11pt]{article}

    
    
    \usepackage[T1]{fontenc}
    % Nicer default font (+ math font) than Computer Modern for most use cases
    \usepackage{mathpazo}

    % Basic figure setup, for now with no caption control since it's done
    % automatically by Pandoc (which extracts ![](path) syntax from Markdown).
    \usepackage{graphicx}
    % We will generate all images so they have a width \maxwidth. This means
    % that they will get their normal width if they fit onto the page, but
    % are scaled down if they would overflow the margins.
    \makeatletter
    \def\maxwidth{\ifdim\Gin@nat@width>\linewidth\linewidth
    \else\Gin@nat@width\fi}
    \makeatother
    \let\Oldincludegraphics\includegraphics
    % Set max figure width to be 80% of text width, for now hardcoded.
    \renewcommand{\includegraphics}[1]{\Oldincludegraphics[width=.8\maxwidth]{#1}}
    % Ensure that by default, figures have no caption (until we provide a
    % proper Figure object with a Caption API and a way to capture that
    % in the conversion process - todo).
    \usepackage{caption}
    \DeclareCaptionLabelFormat{nolabel}{}
    \captionsetup{labelformat=nolabel}

    \usepackage{adjustbox} % Used to constrain images to a maximum size 
    \usepackage{xcolor} % Allow colors to be defined
    \usepackage{enumerate} % Needed for markdown enumerations to work
    \usepackage{geometry} % Used to adjust the document margins
    \usepackage{amsmath} % Equations
    \usepackage{amssymb} % Equations
    \usepackage{textcomp} % defines textquotesingle
    % Hack from http://tex.stackexchange.com/a/47451/13684:
    \AtBeginDocument{%
        \def\PYZsq{\textquotesingle}% Upright quotes in Pygmentized code
    }
    \usepackage{upquote} % Upright quotes for verbatim code
    \usepackage{eurosym} % defines \euro
    \usepackage[mathletters]{ucs} % Extended unicode (utf-8) support
    \usepackage[utf8x]{inputenc} % Allow utf-8 characters in the tex document
    \usepackage{fancyvrb} % verbatim replacement that allows latex
    \usepackage{grffile} % extends the file name processing of package graphics 
                         % to support a larger range 
    % The hyperref package gives us a pdf with properly built
    % internal navigation ('pdf bookmarks' for the table of contents,
    % internal cross-reference links, web links for URLs, etc.)
    \usepackage{hyperref}
    \usepackage{longtable} % longtable support required by pandoc >1.10
    \usepackage{booktabs}  % table support for pandoc > 1.12.2
    \usepackage[inline]{enumitem} % IRkernel/repr support (it uses the enumerate* environment)
    \usepackage[normalem]{ulem} % ulem is needed to support strikethroughs (\sout)
                                % normalem makes italics be italics, not underlines
    

    
    
    % Colors for the hyperref package
    \definecolor{urlcolor}{rgb}{0,.145,.698}
    \definecolor{linkcolor}{rgb}{.71,0.21,0.01}
    \definecolor{citecolor}{rgb}{.12,.54,.11}

    % ANSI colors
    \definecolor{ansi-black}{HTML}{3E424D}
    \definecolor{ansi-black-intense}{HTML}{282C36}
    \definecolor{ansi-red}{HTML}{E75C58}
    \definecolor{ansi-red-intense}{HTML}{B22B31}
    \definecolor{ansi-green}{HTML}{00A250}
    \definecolor{ansi-green-intense}{HTML}{007427}
    \definecolor{ansi-yellow}{HTML}{DDB62B}
    \definecolor{ansi-yellow-intense}{HTML}{B27D12}
    \definecolor{ansi-blue}{HTML}{208FFB}
    \definecolor{ansi-blue-intense}{HTML}{0065CA}
    \definecolor{ansi-magenta}{HTML}{D160C4}
    \definecolor{ansi-magenta-intense}{HTML}{A03196}
    \definecolor{ansi-cyan}{HTML}{60C6C8}
    \definecolor{ansi-cyan-intense}{HTML}{258F8F}
    \definecolor{ansi-white}{HTML}{C5C1B4}
    \definecolor{ansi-white-intense}{HTML}{A1A6B2}

    % commands and environments needed by pandoc snippets
    % extracted from the output of `pandoc -s`
    \providecommand{\tightlist}{%
      \setlength{\itemsep}{0pt}\setlength{\parskip}{0pt}}
    \DefineVerbatimEnvironment{Highlighting}{Verbatim}{commandchars=\\\{\}}
    % Add ',fontsize=\small' for more characters per line
    \newenvironment{Shaded}{}{}
    \newcommand{\KeywordTok}[1]{\textcolor[rgb]{0.00,0.44,0.13}{\textbf{{#1}}}}
    \newcommand{\DataTypeTok}[1]{\textcolor[rgb]{0.56,0.13,0.00}{{#1}}}
    \newcommand{\DecValTok}[1]{\textcolor[rgb]{0.25,0.63,0.44}{{#1}}}
    \newcommand{\BaseNTok}[1]{\textcolor[rgb]{0.25,0.63,0.44}{{#1}}}
    \newcommand{\FloatTok}[1]{\textcolor[rgb]{0.25,0.63,0.44}{{#1}}}
    \newcommand{\CharTok}[1]{\textcolor[rgb]{0.25,0.44,0.63}{{#1}}}
    \newcommand{\StringTok}[1]{\textcolor[rgb]{0.25,0.44,0.63}{{#1}}}
    \newcommand{\CommentTok}[1]{\textcolor[rgb]{0.38,0.63,0.69}{\textit{{#1}}}}
    \newcommand{\OtherTok}[1]{\textcolor[rgb]{0.00,0.44,0.13}{{#1}}}
    \newcommand{\AlertTok}[1]{\textcolor[rgb]{1.00,0.00,0.00}{\textbf{{#1}}}}
    \newcommand{\FunctionTok}[1]{\textcolor[rgb]{0.02,0.16,0.49}{{#1}}}
    \newcommand{\RegionMarkerTok}[1]{{#1}}
    \newcommand{\ErrorTok}[1]{\textcolor[rgb]{1.00,0.00,0.00}{\textbf{{#1}}}}
    \newcommand{\NormalTok}[1]{{#1}}
    
    % Additional commands for more recent versions of Pandoc
    \newcommand{\ConstantTok}[1]{\textcolor[rgb]{0.53,0.00,0.00}{{#1}}}
    \newcommand{\SpecialCharTok}[1]{\textcolor[rgb]{0.25,0.44,0.63}{{#1}}}
    \newcommand{\VerbatimStringTok}[1]{\textcolor[rgb]{0.25,0.44,0.63}{{#1}}}
    \newcommand{\SpecialStringTok}[1]{\textcolor[rgb]{0.73,0.40,0.53}{{#1}}}
    \newcommand{\ImportTok}[1]{{#1}}
    \newcommand{\DocumentationTok}[1]{\textcolor[rgb]{0.73,0.13,0.13}{\textit{{#1}}}}
    \newcommand{\AnnotationTok}[1]{\textcolor[rgb]{0.38,0.63,0.69}{\textbf{\textit{{#1}}}}}
    \newcommand{\CommentVarTok}[1]{\textcolor[rgb]{0.38,0.63,0.69}{\textbf{\textit{{#1}}}}}
    \newcommand{\VariableTok}[1]{\textcolor[rgb]{0.10,0.09,0.49}{{#1}}}
    \newcommand{\ControlFlowTok}[1]{\textcolor[rgb]{0.00,0.44,0.13}{\textbf{{#1}}}}
    \newcommand{\OperatorTok}[1]{\textcolor[rgb]{0.40,0.40,0.40}{{#1}}}
    \newcommand{\BuiltInTok}[1]{{#1}}
    \newcommand{\ExtensionTok}[1]{{#1}}
    \newcommand{\PreprocessorTok}[1]{\textcolor[rgb]{0.74,0.48,0.00}{{#1}}}
    \newcommand{\AttributeTok}[1]{\textcolor[rgb]{0.49,0.56,0.16}{{#1}}}
    \newcommand{\InformationTok}[1]{\textcolor[rgb]{0.38,0.63,0.69}{\textbf{\textit{{#1}}}}}
    \newcommand{\WarningTok}[1]{\textcolor[rgb]{0.38,0.63,0.69}{\textbf{\textit{{#1}}}}}
    
    
    % Define a nice break command that doesn't care if a line doesn't already
    % exist.
    \def\br{\hspace*{\fill} \\* }
    % Math Jax compatability definitions
    \def\gt{>}
    \def\lt{<}
    % Document parameters
    \title{E6156-f2018-Lecture-1}
    
    
    

    % Pygments definitions
    
\makeatletter
\def\PY@reset{\let\PY@it=\relax \let\PY@bf=\relax%
    \let\PY@ul=\relax \let\PY@tc=\relax%
    \let\PY@bc=\relax \let\PY@ff=\relax}
\def\PY@tok#1{\csname PY@tok@#1\endcsname}
\def\PY@toks#1+{\ifx\relax#1\empty\else%
    \PY@tok{#1}\expandafter\PY@toks\fi}
\def\PY@do#1{\PY@bc{\PY@tc{\PY@ul{%
    \PY@it{\PY@bf{\PY@ff{#1}}}}}}}
\def\PY#1#2{\PY@reset\PY@toks#1+\relax+\PY@do{#2}}

\expandafter\def\csname PY@tok@w\endcsname{\def\PY@tc##1{\textcolor[rgb]{0.73,0.73,0.73}{##1}}}
\expandafter\def\csname PY@tok@c\endcsname{\let\PY@it=\textit\def\PY@tc##1{\textcolor[rgb]{0.25,0.50,0.50}{##1}}}
\expandafter\def\csname PY@tok@cp\endcsname{\def\PY@tc##1{\textcolor[rgb]{0.74,0.48,0.00}{##1}}}
\expandafter\def\csname PY@tok@k\endcsname{\let\PY@bf=\textbf\def\PY@tc##1{\textcolor[rgb]{0.00,0.50,0.00}{##1}}}
\expandafter\def\csname PY@tok@kp\endcsname{\def\PY@tc##1{\textcolor[rgb]{0.00,0.50,0.00}{##1}}}
\expandafter\def\csname PY@tok@kt\endcsname{\def\PY@tc##1{\textcolor[rgb]{0.69,0.00,0.25}{##1}}}
\expandafter\def\csname PY@tok@o\endcsname{\def\PY@tc##1{\textcolor[rgb]{0.40,0.40,0.40}{##1}}}
\expandafter\def\csname PY@tok@ow\endcsname{\let\PY@bf=\textbf\def\PY@tc##1{\textcolor[rgb]{0.67,0.13,1.00}{##1}}}
\expandafter\def\csname PY@tok@nb\endcsname{\def\PY@tc##1{\textcolor[rgb]{0.00,0.50,0.00}{##1}}}
\expandafter\def\csname PY@tok@nf\endcsname{\def\PY@tc##1{\textcolor[rgb]{0.00,0.00,1.00}{##1}}}
\expandafter\def\csname PY@tok@nc\endcsname{\let\PY@bf=\textbf\def\PY@tc##1{\textcolor[rgb]{0.00,0.00,1.00}{##1}}}
\expandafter\def\csname PY@tok@nn\endcsname{\let\PY@bf=\textbf\def\PY@tc##1{\textcolor[rgb]{0.00,0.00,1.00}{##1}}}
\expandafter\def\csname PY@tok@ne\endcsname{\let\PY@bf=\textbf\def\PY@tc##1{\textcolor[rgb]{0.82,0.25,0.23}{##1}}}
\expandafter\def\csname PY@tok@nv\endcsname{\def\PY@tc##1{\textcolor[rgb]{0.10,0.09,0.49}{##1}}}
\expandafter\def\csname PY@tok@no\endcsname{\def\PY@tc##1{\textcolor[rgb]{0.53,0.00,0.00}{##1}}}
\expandafter\def\csname PY@tok@nl\endcsname{\def\PY@tc##1{\textcolor[rgb]{0.63,0.63,0.00}{##1}}}
\expandafter\def\csname PY@tok@ni\endcsname{\let\PY@bf=\textbf\def\PY@tc##1{\textcolor[rgb]{0.60,0.60,0.60}{##1}}}
\expandafter\def\csname PY@tok@na\endcsname{\def\PY@tc##1{\textcolor[rgb]{0.49,0.56,0.16}{##1}}}
\expandafter\def\csname PY@tok@nt\endcsname{\let\PY@bf=\textbf\def\PY@tc##1{\textcolor[rgb]{0.00,0.50,0.00}{##1}}}
\expandafter\def\csname PY@tok@nd\endcsname{\def\PY@tc##1{\textcolor[rgb]{0.67,0.13,1.00}{##1}}}
\expandafter\def\csname PY@tok@s\endcsname{\def\PY@tc##1{\textcolor[rgb]{0.73,0.13,0.13}{##1}}}
\expandafter\def\csname PY@tok@sd\endcsname{\let\PY@it=\textit\def\PY@tc##1{\textcolor[rgb]{0.73,0.13,0.13}{##1}}}
\expandafter\def\csname PY@tok@si\endcsname{\let\PY@bf=\textbf\def\PY@tc##1{\textcolor[rgb]{0.73,0.40,0.53}{##1}}}
\expandafter\def\csname PY@tok@se\endcsname{\let\PY@bf=\textbf\def\PY@tc##1{\textcolor[rgb]{0.73,0.40,0.13}{##1}}}
\expandafter\def\csname PY@tok@sr\endcsname{\def\PY@tc##1{\textcolor[rgb]{0.73,0.40,0.53}{##1}}}
\expandafter\def\csname PY@tok@ss\endcsname{\def\PY@tc##1{\textcolor[rgb]{0.10,0.09,0.49}{##1}}}
\expandafter\def\csname PY@tok@sx\endcsname{\def\PY@tc##1{\textcolor[rgb]{0.00,0.50,0.00}{##1}}}
\expandafter\def\csname PY@tok@m\endcsname{\def\PY@tc##1{\textcolor[rgb]{0.40,0.40,0.40}{##1}}}
\expandafter\def\csname PY@tok@gh\endcsname{\let\PY@bf=\textbf\def\PY@tc##1{\textcolor[rgb]{0.00,0.00,0.50}{##1}}}
\expandafter\def\csname PY@tok@gu\endcsname{\let\PY@bf=\textbf\def\PY@tc##1{\textcolor[rgb]{0.50,0.00,0.50}{##1}}}
\expandafter\def\csname PY@tok@gd\endcsname{\def\PY@tc##1{\textcolor[rgb]{0.63,0.00,0.00}{##1}}}
\expandafter\def\csname PY@tok@gi\endcsname{\def\PY@tc##1{\textcolor[rgb]{0.00,0.63,0.00}{##1}}}
\expandafter\def\csname PY@tok@gr\endcsname{\def\PY@tc##1{\textcolor[rgb]{1.00,0.00,0.00}{##1}}}
\expandafter\def\csname PY@tok@ge\endcsname{\let\PY@it=\textit}
\expandafter\def\csname PY@tok@gs\endcsname{\let\PY@bf=\textbf}
\expandafter\def\csname PY@tok@gp\endcsname{\let\PY@bf=\textbf\def\PY@tc##1{\textcolor[rgb]{0.00,0.00,0.50}{##1}}}
\expandafter\def\csname PY@tok@go\endcsname{\def\PY@tc##1{\textcolor[rgb]{0.53,0.53,0.53}{##1}}}
\expandafter\def\csname PY@tok@gt\endcsname{\def\PY@tc##1{\textcolor[rgb]{0.00,0.27,0.87}{##1}}}
\expandafter\def\csname PY@tok@err\endcsname{\def\PY@bc##1{\setlength{\fboxsep}{0pt}\fcolorbox[rgb]{1.00,0.00,0.00}{1,1,1}{\strut ##1}}}
\expandafter\def\csname PY@tok@kc\endcsname{\let\PY@bf=\textbf\def\PY@tc##1{\textcolor[rgb]{0.00,0.50,0.00}{##1}}}
\expandafter\def\csname PY@tok@kd\endcsname{\let\PY@bf=\textbf\def\PY@tc##1{\textcolor[rgb]{0.00,0.50,0.00}{##1}}}
\expandafter\def\csname PY@tok@kn\endcsname{\let\PY@bf=\textbf\def\PY@tc##1{\textcolor[rgb]{0.00,0.50,0.00}{##1}}}
\expandafter\def\csname PY@tok@kr\endcsname{\let\PY@bf=\textbf\def\PY@tc##1{\textcolor[rgb]{0.00,0.50,0.00}{##1}}}
\expandafter\def\csname PY@tok@bp\endcsname{\def\PY@tc##1{\textcolor[rgb]{0.00,0.50,0.00}{##1}}}
\expandafter\def\csname PY@tok@fm\endcsname{\def\PY@tc##1{\textcolor[rgb]{0.00,0.00,1.00}{##1}}}
\expandafter\def\csname PY@tok@vc\endcsname{\def\PY@tc##1{\textcolor[rgb]{0.10,0.09,0.49}{##1}}}
\expandafter\def\csname PY@tok@vg\endcsname{\def\PY@tc##1{\textcolor[rgb]{0.10,0.09,0.49}{##1}}}
\expandafter\def\csname PY@tok@vi\endcsname{\def\PY@tc##1{\textcolor[rgb]{0.10,0.09,0.49}{##1}}}
\expandafter\def\csname PY@tok@vm\endcsname{\def\PY@tc##1{\textcolor[rgb]{0.10,0.09,0.49}{##1}}}
\expandafter\def\csname PY@tok@sa\endcsname{\def\PY@tc##1{\textcolor[rgb]{0.73,0.13,0.13}{##1}}}
\expandafter\def\csname PY@tok@sb\endcsname{\def\PY@tc##1{\textcolor[rgb]{0.73,0.13,0.13}{##1}}}
\expandafter\def\csname PY@tok@sc\endcsname{\def\PY@tc##1{\textcolor[rgb]{0.73,0.13,0.13}{##1}}}
\expandafter\def\csname PY@tok@dl\endcsname{\def\PY@tc##1{\textcolor[rgb]{0.73,0.13,0.13}{##1}}}
\expandafter\def\csname PY@tok@s2\endcsname{\def\PY@tc##1{\textcolor[rgb]{0.73,0.13,0.13}{##1}}}
\expandafter\def\csname PY@tok@sh\endcsname{\def\PY@tc##1{\textcolor[rgb]{0.73,0.13,0.13}{##1}}}
\expandafter\def\csname PY@tok@s1\endcsname{\def\PY@tc##1{\textcolor[rgb]{0.73,0.13,0.13}{##1}}}
\expandafter\def\csname PY@tok@mb\endcsname{\def\PY@tc##1{\textcolor[rgb]{0.40,0.40,0.40}{##1}}}
\expandafter\def\csname PY@tok@mf\endcsname{\def\PY@tc##1{\textcolor[rgb]{0.40,0.40,0.40}{##1}}}
\expandafter\def\csname PY@tok@mh\endcsname{\def\PY@tc##1{\textcolor[rgb]{0.40,0.40,0.40}{##1}}}
\expandafter\def\csname PY@tok@mi\endcsname{\def\PY@tc##1{\textcolor[rgb]{0.40,0.40,0.40}{##1}}}
\expandafter\def\csname PY@tok@il\endcsname{\def\PY@tc##1{\textcolor[rgb]{0.40,0.40,0.40}{##1}}}
\expandafter\def\csname PY@tok@mo\endcsname{\def\PY@tc##1{\textcolor[rgb]{0.40,0.40,0.40}{##1}}}
\expandafter\def\csname PY@tok@ch\endcsname{\let\PY@it=\textit\def\PY@tc##1{\textcolor[rgb]{0.25,0.50,0.50}{##1}}}
\expandafter\def\csname PY@tok@cm\endcsname{\let\PY@it=\textit\def\PY@tc##1{\textcolor[rgb]{0.25,0.50,0.50}{##1}}}
\expandafter\def\csname PY@tok@cpf\endcsname{\let\PY@it=\textit\def\PY@tc##1{\textcolor[rgb]{0.25,0.50,0.50}{##1}}}
\expandafter\def\csname PY@tok@c1\endcsname{\let\PY@it=\textit\def\PY@tc##1{\textcolor[rgb]{0.25,0.50,0.50}{##1}}}
\expandafter\def\csname PY@tok@cs\endcsname{\let\PY@it=\textit\def\PY@tc##1{\textcolor[rgb]{0.25,0.50,0.50}{##1}}}

\def\PYZbs{\char`\\}
\def\PYZus{\char`\_}
\def\PYZob{\char`\{}
\def\PYZcb{\char`\}}
\def\PYZca{\char`\^}
\def\PYZam{\char`\&}
\def\PYZlt{\char`\<}
\def\PYZgt{\char`\>}
\def\PYZsh{\char`\#}
\def\PYZpc{\char`\%}
\def\PYZdl{\char`\$}
\def\PYZhy{\char`\-}
\def\PYZsq{\char`\'}
\def\PYZdq{\char`\"}
\def\PYZti{\char`\~}
% for compatibility with earlier versions
\def\PYZat{@}
\def\PYZlb{[}
\def\PYZrb{]}
\makeatother


    % Exact colors from NB
    \definecolor{incolor}{rgb}{0.0, 0.0, 0.5}
    \definecolor{outcolor}{rgb}{0.545, 0.0, 0.0}



    
    % Prevent overflowing lines due to hard-to-break entities
    \sloppy 
    % Setup hyperref package
    \hypersetup{
      breaklinks=true,  % so long urls are correctly broken across lines
      colorlinks=true,
      urlcolor=urlcolor,
      linkcolor=linkcolor,
      citecolor=citecolor,
      }
    % Slightly bigger margins than the latex defaults
    
    \geometry{verbose,tmargin=1in,bmargin=1in,lmargin=1in,rmargin=1in}
    
    

    \begin{document}
    
    
    \maketitle
    
    

    
    \section{\texorpdfstring{Topics in SW Engineering: Cloud and
Microservice ApplicationsLecture 1: Introduction and
Overview}{Topics in SW Engineering: Cloud and Microservice Applications Lecture 1: Introduction and Overview}}\label{topics-in-sw-engineering-cloud-and-microservice-applications-lecture-1-introduction-and-overview}

    \subsection{About your instructor}\label{about-your-instructor}

\begin{itemize}
\tightlist
\item
  \emph{Professor of Professional Practice,} Dept. of Computer Science.

  \begin{itemize}
  \tightlist
  \item
    624 Shapiro/CEPSR
  \item
    dff9@columbia.edu
  \end{itemize}
\item
  Academic experience

  \begin{itemize}
  \tightlist
  \item
    Ph.D. in Computer Science, Columbia University, 1989
  \item
    Joined Columbia as first full time \emph{Professor of Professional
    Practice}, 01-Jan-2018
  \item
    8 semesters as an adjunct professor teaching

    \begin{itemize}
    \tightlist
    \item
      \emph{E6998: Topics in Computer Science}

      \begin{itemize}
      \tightlist
      \item
        Cloud Computing
      \item
        Web and Internet Application Development
      \item
        Web Application Servers and Applications
      \item
        Microservices
      \end{itemize}
    \item
      \emph{W4111 - Introduction to Databases}
    \item
      \emph{E1006 - Introduction to Computing for Engineers and Applied
      Scientists using Python}
    \end{itemize}
  \end{itemize}
\item
  35 years industry experience

  \begin{itemize}
  \tightlist
  \item
    \href{https://en.wikipedia.org/wiki/IBM_Fellow}{IBM Fellow}, Chief
    Architect for {[}IBM Software
    Group{]}(https://en.wikipedia.org/wiki/IBM\_Software\_Group\_(SWG)
  \item
    Microsoft Technical Fellow
  \item
    Executive Vice President, Chief Technology Officer,
    \href{https://www.ca.com/us.html}{CA Technologies}
  \item
    Vice President, CTO, Senior Fellow,
    \href{https://en.wikipedia.org/wiki/Dell_Software}{Dell Software
    Group}
  \item
    Co-Founder and CTO, \href{https://seekatv.com/}{Seeka TV}
  \end{itemize}
\item
  Publications

  \begin{itemize}
  \tightlist
  \item
    Approximately 60 technical publications.
  \item
    Authored, co-authored several standards in web applications and web
    services.
  \item
    12 patents.
  \end{itemize}
\item
  Personal and hobbies

  \begin{itemize}
  \tightlist
  \item
    Two amazing daughters (One is Barnard student. One is a junior in
    high school).
  \item
    Interested in languages. Speak Spanish reasonably well and trying to
    learn Arabic.
  \item
    Black Belt in Kenpo Karate. Taken Krav Maga.
  \item
    Amateur astronomer.
  \item
    Road bicycling.
  \item
    Officer in the New York Guard.
  \end{itemize}
\end{itemize}

\begin{longtable}[]{@{}c@{}}
\toprule
\tabularnewline
\midrule
\endhead
\textbf{About Me}\tabularnewline
\bottomrule
\end{longtable}

    \subsection{About this Course}\label{about-this-course}

\subsubsection{Objectives}\label{objectives}

\begin{itemize}
\item
  We will form small teams and build a simple, realistic
  microservice/cloud application.
\item
  Course objectives:

  \begin{itemize}
  \tightlist
  \item
    Practical experience with modern technology for building solutions.
  \item
    Experience with patterns and best practices. My experience has been
    that most students write "crappy code."
  \item
    Prepare you for internships and jobs.
  \item
    Have seriously cool stuff to put on your resume and discuss on
    interviews.
  \end{itemize}
\end{itemize}

\subsubsection{\texorpdfstring{Core Topics \(-\)
Overview}{Core Topics - Overview}}\label{core-topics---overview}

\paragraph{Microservice}\label{microservice}

"A 'microservice' is a software development technique---a variant of the
service-oriented architecture (SOA) architectural style that structures
an application as a collection of loosely coupled services. In a
microservices architecture, services are fine-grained and the protocols
are lightweight. The benefit of decomposing an application into
different smaller services is that it improves modularity and makes the
application easier to understand, develop, test, and more resilient to
architecture erosion. It parallelizes development by enabling small
autonomous teams to develop, deploy and scale their respective services
independently. It also allows the architecture of an individual service
to emerge through continuous refactoring. Microservices-based
architectures enable continuous delivery and deployment."
(https://en.wikipedia.org/wiki/Microservices)

\textbf{Note:} {I often cut and paste definitions from Wikipedia.
Wikipedia is not normative and is one of many views on any topic. The
pages are a good overview and launching point for more details. Plus, if
you can use Github for code and solutions, I can cut and paste text.}

Why microservices? - Will cover the technical benefits in next lecture.

\begin{itemize}
\tightlist
\item
  Hot, foundational technology and mastering it benefits your career.
\end{itemize}

\begin{longtable}[]{@{}c@{}}
\toprule
\tabularnewline
\midrule
\endhead
\href{https://www.marketresearchfuture.com/reports/microservices-architecture-market-3149}{Microservices
Architecture Market Research Report-Global Forecast 2023}\tabularnewline
\bottomrule
\end{longtable}

\begin{longtable}[]{@{}c@{}}
\toprule
\tabularnewline
\midrule
\endhead
\href{https://www.nginx.com/resources/library/app-dev-survey/}{Microservice
Adoption 2015}\tabularnewline
\bottomrule
\end{longtable}

    \subsubsection{Cloud Application}\label{cloud-application}

\textbf{Concept}

    "Cloud computing is an information technology (IT) paradigm that enables
ubiquitous access to shared pools of configurable system resources and
higher-level services that can be rapidly provisioned with minimal
management effort, often over the Internet. Cloud computing relies on
sharing of resources to achieve coherence and economies of scale,
similar to a public utility."
(https://en.wikipedia.org/wiki/Cloud\_computing)

\begin{longtable}[]{@{}c@{}}
\toprule
\tabularnewline
\midrule
\endhead
\href{https://en.wikipedia.org/wiki/Cloud_computing}{Cloud
Computing}\tabularnewline
\bottomrule
\end{longtable}

\textbf{NIST Conceptual Model}

\begin{itemize}
\item
  National Institute of Standards and Technology (NIST) defined a
  conceptual model for
  \href{https://ws680.nist.gov/publication/get_pdf.cfm?pub_id=909505}{cloud
  computing.}
\item
  Somewhat date and conceptual, which means the ideas matter but no one
  directly realizes the architecture.
\end{itemize}

\begin{longtable}[]{@{}c@{}}
\toprule
\tabularnewline
\midrule
\endhead
\textbf{NIST Cloud Concetual Model}\tabularnewline
\bottomrule
\end{longtable}

NIST Terminology - \emph{Infrastructure-as-a-Service:} "The capability
provided to the consumer is to provision processing, storage, networks,
and other fundamental computing resources where the consumer is able to
deploy and run arbitrary software, which can include operating systems
and applications. The consumer does not manage or control the underlying
cloud infrastructure but has control over operating systems, storage,
and deployed applications; and possibly limited control of select
networking components (e.g., host firewalls)." -
\emph{Platform-as-a-Service:} "The capability provided to the consumer
is to deploy onto the cloud infrastructure consumer-created or acquired
applications created using programming languages, libraries, services,
and tools supported by the provider.3 The consumer does not manage or
control the underlying cloud infrastructure including network, servers,
operating systems, or storage, but has control over the deployed
applications and possibly configuration settings for the
application-hosting environment." - \emph{Software-as-a-Service:} The
capability provided to the consumer is to use the provider's
applications running on a cloud infrastructure2 . The applications are
accessible from various client devices through either a thin client
interface, such as a web browser (e.g., web-based email), or a program
interface. The consumer does not manage or control the underlying cloud
infrastructure including network, servers, operating systems, storage,
or even individual application capabilities, with the possible exception
of limited userspecific application configuration settings.

\begin{longtable}[]{@{}c@{}}
\toprule
\tabularnewline
\midrule
\endhead
\href{https://slideplayer.com/slide/4582525/}{Cloud Computing
Layers}\tabularnewline
\bottomrule
\end{longtable}

    Some additional concepts: - \emph{Mobile "backend" as a service
(MBaaS):} In the mobile "backend" as a service (m) model, also known as
backend as a service (BaaS), web app and mobile app developers are
provided with a way to link their applications to cloud storage and
cloud computing services with application programming interfaces (APIs)
exposed to their applications and custom software development kits
(SDKs). Services include user management, push notifications,
integration with social networking services and more. This is a
relatively recent model in cloud computing,{[}74{]} with most BaaS
startups dating from 2011 or later but trends indicate that these
services are gaining significant mainstream traction with enterprise
consumers.

\begin{itemize}
\item
  \emph{Serverless computing:} Serverless computing is a cloud computing
  code execution model in which the cloud provider fully manages
  starting and stopping virtual machines as necessary to serve requests,
  and requests are billed by an abstract measure of the resources
  required to satisfy the request, rather than per virtual machine, per
  hour. Despite the name, it does not actually involve running code
  without servers. Serverless computing is so named because the business
  or person that owns the system does not have to purchase, rent or
  provision servers or virtual machines for the back-end code to run on.
\item
  \emph{Function as a service (FaaS):} Function as a service (FaaS) is a
  service-hosted remote procedure call that leverages serverless
  computing to enable the deployment of individual functions in the
  cloud that run in response to events. FaaS is included under the
  broader term serverless computing, but the terms may also be used
  interchangeably.
\end{itemize}

     

    \subsubsection{How Does this Come
Together?}\label{how-does-this-come-together}

\begin{itemize}
\tightlist
\item
  IaaS provides the "hardware" in shared, scalable, elastic model
  consumable in chunks, usually \emph{virtual machines} or
  \emph{containers.}
\item
  PaaS is prebuilt, prepackaged, application enablement and delivery
  software for building microservices.
\item
  IaaS provides "right size" independent, virtual machines/containers to
  execute the microservice and supporting SW.
\item
  SaaS is a complete, customizable, configurable solution to a problem,
  with an internal implementation that is a set of cooperating
  microservices.
\item
  SaaS offers a user interface and APIs. New applications built as
  microservices on PaaS rely heavily on calling SaaS APIs and APIs into
  other forms of *aaS.
\end{itemize}

    \subsubsection{So What are we Going to
Do?}\label{so-what-are-we-going-to-do}

\textbf{Course Work}

\begin{itemize}
\item
  Build a simple multi-tenant cloud application:

  \begin{itemize}
  \tightlist
  \item
    A set of microservices.
  \item
    Using elements of cloud computing:

    \begin{itemize}
    \tightlist
    \item
      PaaS
    \item
      FaaS
    \item
      Loggin
    \item
      Serverless
    \item
      Cloud Database(s)
    \item
      Security: Authentication, Authorization and Federation.
    \item
      Message Queues, Events and Event-Driven-Architecture
    \item
      Simple workflow and service orchestration.
    \item
      Call cloud business APIs.
    \item
      Other cool stuff.
    \end{itemize}
  \end{itemize}
\item
  Modeled on my startup's application, but you will define:

  \begin{itemize}
  \tightlist
  \item
    The application.
  \item
    Business scenario and value.
  \item
    Features and functions.
  \end{itemize}
\item
  Periodically demo, present and have an architecture review.
\item
  Sort of like a small startup, only with a bit more formal reviews.
\end{itemize}

\textbf{Sparq/Seeka Demo and Overview}

\begin{itemize}
\item
  Demo
\item
  Multi-tenancy
\item
  Architecture
\end{itemize}

    \begin{longtable}[]{@{}c@{}}
\toprule
\tabularnewline
\midrule
\endhead
\textbf{Sparq/Seeka}\tabularnewline
\bottomrule
\end{longtable}

    \begin{longtable}[]{@{}c@{}}
\toprule
\tabularnewline
\midrule
\endhead
\textbf{Sparq/Seeka}\tabularnewline
\bottomrule
\end{longtable}

    \textbf{Note:} - Amazon Web Services will be the realization of most
technical concepts (e.g. serverless, pub/sub).

\begin{itemize}
\item
  AWS is the dominant public cloud platform, but other vendors are
  becoming significant.
\item
  Concepts are what is important; you can master any realization once
  you understand the concepts.
\item
  Will try to factor in other cloud providers, e.g. Google.
\end{itemize}

    \begin{longtable}[]{@{}c@{}}
\toprule
\tabularnewline
\midrule
\endhead
\href{https://www.rightscale.com/blog/cloud-industry-insights/cloud-computing-trends-2018-state-cloud-survey}{Cloud
Provider Adoption}\tabularnewline
\bottomrule
\end{longtable}

    \subsubsection{Course Format, Grading and
Environment}\label{course-format-grading-and-environment}

\begin{itemize}
\tightlist
\item
  Lectures: Friday, 1:10pm to 3:40pm; 603 Hamilton Hall

  \begin{itemize}
  \tightlist
  \item
    1:10 - 3:00 will be lecture on material.
  \item
    3:00 - 3:40 will be optional discussion, recitation, etc.
  \end{itemize}
\item
  Office Hours:

  \begin{itemize}
  \tightlist
  \item
    Thursday, 8:00 AM to 1:00 PM
  \item
    By appointment, as needed, as available. I typically post on Piazza
    when I will have extra availability.
  \end{itemize}
\item
  Course Material:

  \begin{itemize}
  \tightlist
  \item
    No textbook

    \begin{itemize}
    \tightlist
    \item
      Textbooks become out of date in this rapidly changing area.
    \item
      Material would span several books.
    \end{itemize}
  \item
    Lecture notes and code samples
    \href{https://github.com/donald-f-ferguson/E6156f18}{on GitHub}.
    Sample code is mostly JavaScript and NodeJS.
  \item
    References to web documents and tutorials.
  \end{itemize}
\item
  Grading:

  \begin{itemize}
  \tightlist
  \item
    Based on final project, including project presentation. 4-5 person
    teams.
  \item
    Mandatory interim checkpoints and presentations, approximately every
    two weeks.
  \item
    Meeting instructor defined project requirements is an "A."
  \end{itemize}
\item
  Environment:

  \begin{itemize}
  \tightlist
  \item
    Required: AWS (team) free
    \href{https://aws.amazon.com/free/?sc_channel=PS\&sc_campaign=acquisition_US\&sc_publisher=google\&sc_medium=ACQ-P\%7CPS-GO\%7CBrand\%7CSU\%7CCore\%7CCore\%7CUS\%7CEN\%7CText\&sc_content=Brand_Free_e\&sc_detail=amazon\%20free\%20tier\&sc_category=core\&sc_segment=280392800801\&sc_matchtype=e\&sc_country=US\&sc_kwcid=AL!4422!3!280392800801!e!!g!!amazon\%20free\%20tier\&s_kwcid=AL!4422!3!280392800801!e!!g!!amazon\%20free\%20tier\&ef_id=WKzT_wAAAJllO8d4:20180811135240:s}{tier
    account}.
  \item
    Recommended:

    \begin{itemize}
    \tightlist
    \item
      WebStorm (free student licenses
      https://www.jetbrains.com/student/).
    \item
      Ananconda for Jupyter Notebooks
      (https://www.anaconda.com/download/\#macos).
    \item
      JavaScript kernel plug-in for Jupyter Notebooks
      (https://github.com/n-riesco/ijavascript).
    \end{itemize}
  \end{itemize}
\end{itemize}

    \subsection{First Microservice}\label{first-microservice}

\begin{itemize}
\item
  \textbf{Enough talking!}
\item
  \textbf{We will go over microservice architecture, benefits, design
  patterns, yada yada ... in future lectures.}
\end{itemize}

\_\_Let's build a microservice. Code rules and slides drool!

\_\_ 

    \subsubsection{Setup}\label{setup}

\begin{enumerate}
\def\labelenumi{\arabic{enumi}.}
\tightlist
\item
  Installed

  \begin{enumerate}
  \def\labelenumii{\arabic{enumii}.}
  \tightlist
  \item
    NodeJS (https://nodejs.org/en/download/).
  \item
    MySQL and MySQL Workbench
    (https://dev.mysql.com/downloads/installer/)
  \item
    WebStorm Integrated Development Environment 
  \end{enumerate}
\item
  Create directory/project

  \begin{enumerate}
  \def\labelenumii{\arabic{enumii}.}
  \tightlist
  \item
    Create directory where you want your code.
  \item
    Install and set up Express -\/- Node.JS Application Framework
    (https://expressjs.com/en/starter/installing.html) 
  \end{enumerate}
\item
  Download static web site template.

  \begin{enumerate}
  \def\labelenumii{\arabic{enumii}.}
  \tightlist
  \item
    I chose
    \href{https://blackrockdigital.github.io/startbootstrap-small-business/}{Bootstrap}
    and prebuilt
    \href{https://blackrockdigital.github.io/startbootstrap-small-business/}{SMB
    Template.}
  \item
    Downloaded and copied to "static content" directory in project.
  \end{enumerate}
\end{enumerate}

    \begin{longtable}[]{@{}c@{}}
\toprule
\tabularnewline
\midrule
\endhead
\textbf{Simple Project Homepage}\tabularnewline
\bottomrule
\end{longtable}

\begin{longtable}[]{@{}c@{}}
\toprule
\tabularnewline
\midrule
\endhead
\_\_Simple Project Structure\tabularnewline
\bottomrule
\end{longtable}

    \subsubsection{Application Design Methodology:
Data-Out/UX-In}\label{application-design-methodology-data-outux-in}

\begin{longtable}[]{@{}c@{}}
\toprule
\tabularnewline
\midrule
\endhead
\textbf{Data-Out/UX-In Design}\tabularnewline
\bottomrule
\end{longtable}

We will use data out for the first step.

    \subsubsection{Data}\label{data}

\begin{longtable}[]{@{}c@{}}
\toprule
\includegraphics{attachment:image.png}\tabularnewline
\midrule
\endhead
\textbf{ER Diagram \(-\) Physical Model}\tabularnewline
\bottomrule
\end{longtable}

     \textbf{Relational Database Table Schema}

\begin{verbatim}
CREATE TABLE `customers` (
  `customers_id` varchar(12) NOT NULL,
  `customers_lastname` varchar(64) NOT NULL,
  `customers_firstname` varchar(64) NOT NULL,
  `customers_email` varchar(128) NOT NULL,
  `customers_status` enum('ACTIVE','PENDING','DELETED') NOT NULL,
  `customers_password` varchar(512) NOT NULL,
  PRIMARY KEY (`customers_id`)
) ENGINE=InnoDB DEFAULT CHARSET=utf8;
\end{verbatim}

    Why a relational database? - I am lazy and can reuse code between both
my classes.

\begin{itemize}
\item
  Very common approach to managing data.
\item
  Good place to start.
\item
  A lot of application scenarios are \emph{microservice-ify my existing
  (relational) data model.}
\end{itemize}

\begin{longtable}[]{@{}c@{}}
\toprule
\includegraphics{attachment:image.png}\tabularnewline
\midrule
\endhead
\href{https://db-engines.com/en/ranking_categories}{DBMS popularity
broken down by database model}\tabularnewline
\bottomrule
\end{longtable}

    \subsubsection{\texorpdfstring{Best Practice \(-\) Data Access
Encapsulation}{Best Practice - Data Access Encapsulation}}\label{best-practice---data-access-encapsulation}

"In computer software, a data access object (DAO) is an object that
provides an abstract interface to some type of database or other
persistence mechanism. By mapping application calls to the persistence
layer, the DAO provides some specific data operations without exposing
details of the database. This isolation supports the single
responsibility principle. It separates what data access the application
needs, in terms of domain-specific objects and data types (the public
interface of the DAO), from how these needs can be satisfied with a
specific DBMS, database schema, etc. (the implementation of the DAO)."
(https://en.wikipedia.org/wiki/Data\_access\_object)

"The single responsibility principle is a computer programming principle
that states that every module or class should have responsibility over a
single part of the functionality provided by the software, and that
responsibility should be entirely encapsulated by the class."
(https://en.wikipedia.org/wiki/Single\_responsibility\_principle)

\begin{longtable}[]{@{}c@{}}
\toprule
\includegraphics{attachment:image.png}\tabularnewline
\midrule
\endhead
\href{http://www.informit.com/articles/article.aspx?p=1398621\&seqNum=3}{Data
Access Object}\tabularnewline
\bottomrule
\end{longtable}

    "..., it is a custom More honour'd in the breach than the observance."
\href{https://en.wikipedia.org/wiki/Hamlet}{Hamlet.}

Comments: - There are many, many variations of this pattern and
supporting frameworks.

\begin{itemize}
\item
  Teams and organizations pick one (or write one) and then apply the
  pattern.
\item
  If you sprinkle data access code in your microservice "business
  logic," I will make changing schema and database backends part of the
  assignment until you learn your lesson.
\item
  I am just going to do a simple implementation.
\end{itemize}

    \begin{Verbatim}[commandchars=\\\{\}]
{\color{incolor}In [{\color{incolor}1}]:} \PY{c+cm}{/**}
        \PY{c+cm}{ * Created by donaldferguson on 8/12/18.}
        \PY{c+cm}{ */}
        \PY{k+kd}{let} \PY{n+nx}{path\PYZus{}base} \PY{o}{=} \PY{l+s+s2}{\PYZdq{}/Users/donaldferguson/Dropbox/ColumbiaCourse/Courses/Fall2018/E6156/Projects/FirstMicroservice/\PYZdq{}}\PY{p}{;}
        \PY{k+kd}{let} \PY{n+nx}{logging} \PY{o}{=} \PY{n+nx}{require}\PY{p}{(}\PY{n+nx}{path\PYZus{}base} \PY{o}{+} \PY{l+s+s2}{\PYZdq{}lib/logging\PYZdq{}}\PY{p}{)}\PY{p}{;}
        \PY{k+kd}{let} \PY{n+nx}{mysql} \PY{o}{=} \PY{n+nx}{require}\PY{p}{(}\PY{l+s+s1}{\PYZsq{}mysql\PYZsq{}}\PY{p}{)}\PY{p}{;}
        
        \PY{k+kr}{class} \PY{n+nx}{Dao} \PY{p}{\PYZob{}}
        
            \PY{n+nx}{constructor}\PY{p}{(}\PY{n+nx}{config}\PY{p}{)} \PY{p}{\PYZob{}}
                \PY{k}{this}\PY{p}{.}\PY{n+nx}{config} \PY{o}{=} \PY{n+nx}{config}\PY{p}{;}
                \PY{k}{this}\PY{p}{.}\PY{n+nx}{db\PYZus{}con} \PY{o}{=} \PY{k+kc}{null}\PY{p}{;}
            \PY{p}{\PYZcb{}}
        
            \PY{n+nx}{sayHello}\PY{p}{(}\PY{p}{)} \PY{p}{\PYZob{}}
                \PY{n+nx}{console}\PY{p}{.}\PY{n+nx}{log}\PY{p}{(}\PY{l+s+s2}{\PYZdq{}config = \PYZdq{}} \PY{o}{+} \PY{n+nx}{JSON}\PY{p}{.}\PY{n+nx}{stringify}\PY{p}{(}\PY{k}{this}\PY{p}{.}\PY{n+nx}{config}\PY{p}{,}\PY{k+kc}{null}\PY{p}{,}\PY{l+m+mi}{4}\PY{p}{)}\PY{p}{)}\PY{p}{;}
            \PY{p}{\PYZcb{}}
        
            \PY{n+nx}{get\PYZus{}db\PYZus{}connection}\PY{p}{(}\PY{p}{)} \PY{p}{\PYZob{}}
                \PY{k}{if} \PY{p}{(}\PY{k}{this}\PY{p}{.}\PY{n+nx}{db\PYZus{}con} \PY{o}{===} \PY{k+kc}{null}\PY{p}{)} \PY{p}{\PYZob{}}
                    \PY{n+nx}{logging}\PY{p}{.}\PY{n+nx}{debug\PYZus{}message}\PY{p}{(}\PY{l+s+s2}{\PYZdq{}db\PYZus{}connect\PYZus{}info = \PYZdq{}}\PY{p}{,} \PY{k}{this}\PY{p}{.}\PY{n+nx}{config}\PY{p}{.}\PY{n+nx}{db\PYZus{}connect\PYZus{}info}\PY{p}{)}
                    \PY{k}{this}\PY{p}{.}\PY{n+nx}{db\PYZus{}con} \PY{o}{=} \PY{n+nx}{mysql}\PY{p}{.}\PY{n+nx}{createConnection}\PY{p}{(}\PY{k}{this}\PY{p}{.}\PY{n+nx}{config}\PY{p}{.}\PY{n+nx}{db\PYZus{}connect\PYZus{}info}\PY{p}{)}\PY{p}{;}
                    \PY{n+nx}{logging}\PY{p}{.}\PY{n+nx}{debug\PYZus{}message}\PY{p}{(}\PY{l+s+s2}{\PYZdq{}Created DB connection\PYZdq{}}\PY{p}{)}\PY{p}{;}
                \PY{p}{\PYZcb{}}
                \PY{k}{return} \PY{k}{this}\PY{p}{.}\PY{n+nx}{db\PYZus{}con}\PY{p}{;}
            \PY{p}{\PYZcb{}}
        
        
            \PY{n+nx}{execute\PYZus{}query}\PY{p}{(}\PY{n+nx}{statement}\PY{p}{,} \PY{n+nx}{values}\PY{p}{,} \PY{n+nx}{context}\PY{p}{)} \PY{p}{\PYZob{}}
        
                \PY{n+nx}{config} \PY{o}{=} \PY{k}{this}\PY{p}{.}\PY{n+nx}{config}\PY{p}{;}
        
                \PY{k}{return}	\PY{k}{new} \PY{n+nb}{Promise}\PY{p}{(}\PY{k+kd}{function}\PY{p}{(}\PY{n+nx}{resolve}\PY{p}{,} \PY{n+nx}{reject}\PY{p}{)} \PY{p}{\PYZob{}}
                    \PY{n+nx}{logging}\PY{p}{.}\PY{n+nx}{debug\PYZus{}message}\PY{p}{(}\PY{l+s+s2}{\PYZdq{}execute\PYZus{}query (enter), statement = \PYZdq{}} \PY{o}{+} \PY{n+nx}{statement}\PY{p}{)}\PY{p}{;}
                    \PY{n+nx}{logging}\PY{p}{.}\PY{n+nx}{debug\PYZus{}message}\PY{p}{(}\PY{l+s+s2}{\PYZdq{}values = \PYZdq{}}\PY{p}{,} \PY{n+nx}{values}\PY{p}{)}\PY{p}{;}
        
                    \PY{k+kd}{var}		\PY{n+nx}{con}		\PY{o}{=}	\PY{n+nx}{mysql}\PY{p}{.}\PY{n+nx}{createConnection}\PY{p}{(}\PY{n+nx}{config}\PY{p}{.}\PY{n+nx}{db\PYZus{}connect\PYZus{}info}\PY{p}{)}\PY{p}{;}
        
                    \PY{n+nx}{logging}\PY{p}{.}\PY{n+nx}{debug\PYZus{}message}\PY{p}{(}\PY{l+s+s2}{\PYZdq{}db.execute\PYZus{}query: After calling con create.\PYZdq{}}\PY{p}{)}\PY{p}{;}
        
                    \PY{n+nx}{con}\PY{p}{.}\PY{n+nx}{connect}\PY{p}{(}\PY{k+kd}{function}\PY{p}{(}\PY{n+nx}{err}\PY{p}{)} \PY{p}{\PYZob{}}
                        \PY{k}{if} \PY{p}{(}\PY{n+nx}{err}\PY{p}{)} \PY{p}{\PYZob{}}
                            \PY{n+nx}{logging}\PY{p}{.}\PY{n+nx}{error\PYZus{}message}\PY{p}{(}\PY{l+s+s2}{\PYZdq{}DB Connect Failed, \PYZdq{}}\PY{p}{)}\PY{p}{;}
                            \PY{n+nx}{reject}\PY{p}{(}\PY{n+nx}{err}\PY{p}{)}\PY{p}{;}			\PY{c+c1}{//	Return the promise error.}
                        \PY{p}{\PYZcb{}}
                        \PY{k}{else} \PY{p}{\PYZob{}}
                            \PY{n+nx}{logging}\PY{p}{.}\PY{n+nx}{debug\PYZus{}message}\PY{p}{(}\PY{l+s+s1}{\PYZsq{}Connected to DB\PYZsq{}}\PY{p}{)}\PY{p}{;}
        
                            \PY{n+nx}{con}\PY{p}{.}\PY{n+nx}{query}\PY{p}{(}\PY{n+nx}{statement}\PY{p}{,} \PY{n+nx}{values}\PY{p}{,} \PY{k+kd}{function}\PY{p}{(}\PY{n+nx}{err}\PY{p}{,}\PY{n+nx}{result}\PY{p}{)}\PY{p}{\PYZob{}}		\PY{c+c1}{//	Execute the query.}
                                \PY{k}{if} \PY{p}{(}\PY{n+nx}{err}\PY{p}{)} \PY{p}{\PYZob{}}
                                    \PY{n+nx}{logging}\PY{p}{.}\PY{n+nx}{debug\PYZus{}message}\PY{p}{(}\PY{l+s+s2}{\PYZdq{}Query failed with error = \PYZdq{}}\PY{p}{,} \PY{n+nx}{err}\PY{p}{)}\PY{p}{;}
                                    \PY{n+nx}{con}\PY{p}{.}\PY{n+nx}{end}\PY{p}{(}\PY{p}{)}\PY{p}{;}
                                    \PY{n+nx}{reject}\PY{p}{(}\PY{n+nx}{err}\PY{p}{)}\PY{p}{;}
                                \PY{p}{\PYZcb{}}
                                \PY{k}{else} \PY{p}{\PYZob{}}
                                    \PY{c+c1}{//logging.debug\PYZus{}message(\PYZdq{}Query statement = \PYZdq{} + statement + \PYZdq{}succeeded, rows = \PYZdq{}, result);}
                                    \PY{n+nx}{logging}\PY{p}{.}\PY{n+nx}{debug\PYZus{}message}\PY{p}{(}\PY{l+s+s2}{\PYZdq{}db.execute\PYZus{}query: After query, result = \PYZdq{}} \PY{o}{+} \PY{n+nx}{JSON}\PY{p}{.}\PY{n+nx}{stringify}\PY{p}{(}\PY{n+nx}{result}\PY{p}{,} \PY{k+kc}{null}\PY{p}{,} \PY{l+m+mi}{2}\PY{p}{)}\PY{p}{)}\PY{p}{;}
                                    \PY{n+nx}{con}\PY{p}{.}\PY{n+nx}{end}\PY{p}{(}\PY{p}{)}\PY{p}{;}
                                    \PY{n+nx}{resolve}\PY{p}{(}\PY{n+nx}{result}\PY{p}{)}\PY{p}{;}						\PY{c+c1}{//	Return the response as promised.}
                                \PY{p}{\PYZcb{}}
                            \PY{p}{\PYZcb{}}\PY{p}{)}\PY{p}{;}
                        \PY{p}{\PYZcb{}}
                    \PY{p}{\PYZcb{}}\PY{p}{)}\PY{p}{;}
                \PY{p}{\PYZcb{}}\PY{p}{)}\PY{p}{;}
            \PY{p}{\PYZcb{}}
        
            \PY{n+nx}{get\PYZus{}by\PYZus{}id}\PY{p}{(}\PY{n+nx}{id}\PY{p}{)} \PY{p}{\PYZob{}}
                \PY{k+kd}{let} \PY{n+nx}{sql\PYZus{}statement} \PY{o}{=} \PY{l+s+s2}{\PYZdq{}SELECT * FROM \PYZdq{}} \PY{o}{+} \PY{k}{this}\PY{p}{.}\PY{n+nx}{config}\PY{p}{.}\PY{n+nx}{data\PYZus{}table} \PY{o}{+}
                    \PY{l+s+s2}{\PYZdq{} where \PYZdq{}} \PY{o}{+} \PY{k}{this}\PY{p}{.}\PY{n+nx}{config}\PY{p}{.}\PY{n+nx}{primary\PYZus{}key} \PY{o}{+} \PY{l+s+s2}{\PYZdq{} = ? \PYZdq{}}\PY{p}{;}
                \PY{k}{return} \PY{k}{this}\PY{p}{.}\PY{n+nx}{execute\PYZus{}query}\PY{p}{(}\PY{n+nx}{sql\PYZus{}statement}\PY{p}{,} \PY{p}{[}\PY{n+nx}{id}\PY{p}{]}\PY{p}{,} \PY{k+kc}{null}\PY{p}{)}\PY{p}{;}
            \PY{p}{\PYZcb{}}
        
        
        \PY{p}{\PYZcb{}}
        
        \PY{k+kd}{let}	\PY{n+nx}{config} \PY{o}{=} \PY{p}{\PYZob{}}
            \PY{n+nx}{db\PYZus{}connect\PYZus{}info}\PY{o}{:} \PY{p}{\PYZob{}}
                \PY{n+nx}{host}\PY{o}{:} \PY{l+s+s2}{\PYZdq{}localhost\PYZdq{}}\PY{p}{,}
                \PY{n+nx}{user}\PY{o}{:} \PY{l+s+s2}{\PYZdq{}dbuser\PYZdq{}}\PY{p}{,}
                \PY{n+nx}{password}\PY{o}{:} \PY{l+s+s2}{\PYZdq{}dbuser\PYZdq{}}
            \PY{p}{\PYZcb{}}\PY{p}{,}
            \PY{n+nx}{data\PYZus{}table}\PY{o}{:} \PY{l+s+s2}{\PYZdq{}e6156.customers\PYZdq{}}\PY{p}{,}
            \PY{n+nx}{primary\PYZus{}key}\PY{o}{:} \PY{l+s+s2}{\PYZdq{}customers\PYZus{}id\PYZdq{}}
        \PY{p}{\PYZcb{}}\PY{p}{;}
        
        \PY{k+kd}{let} \PY{n+nx}{d} \PY{o}{=} \PY{k}{new} \PY{n+nx}{Dao}\PY{p}{(}\PY{n+nx}{config}\PY{p}{)}\PY{p}{;}
        \PY{n+nx}{d}\PY{p}{.}\PY{n+nx}{sayHello}\PY{p}{(}\PY{p}{)}\PY{p}{;}
        \PY{n+nx}{d}\PY{p}{.}\PY{n+nx}{get\PYZus{}by\PYZus{}id}\PY{p}{(}\PY{l+s+s1}{\PYZsq{}df1\PYZsq{}}\PY{p}{)}\PY{p}{.}\PY{n+nx}{then}\PY{p}{(}
            \PY{k+kd}{function}\PY{p}{(}\PY{n+nx}{data}\PY{p}{)} \PY{p}{\PYZob{}}
                \PY{l+s+s2}{\PYZdq{}use strict\PYZdq{}}\PY{p}{;}
                \PY{n+nx}{logging}\PY{p}{.}\PY{n+nx}{debug\PYZus{}message}\PY{p}{(}\PY{l+s+s2}{\PYZdq{}Data = \PYZdq{}}\PY{p}{,} \PY{n+nx}{data}\PY{p}{)}
            \PY{p}{\PYZcb{}}
        \PY{p}{)}\PY{p}{;}
\end{Verbatim}


    \begin{Verbatim}[commandchars=\\\{\}]
config = \{
    "db\_connect\_info": \{
        "host": "localhost",
        "user": "dbuser",
        "password": "dbuser"
    \},
    "data\_table": "e6156.customers",
    "primary\_key": "customers\_id"
\}
execute\_query (enter), statement = SELECT * FROM e6156.customers where customers\_id = ? 
values =  [
  "df1"
]
db.execute\_query: After calling con create.
Connected to DB
db.execute\_query: After query, result = [
  \{
    "customers\_id": "df1",
    "customers\_lastname": "Ferguson",
    "customers\_firstname": "Donald",
    "customers\_email": "dff9@columbia.edu",
    "customers\_status": "ACTIVE",
    "customers\_password": "lakd;lkasddjkl;sADJ"
  \}
]
Data =  [
  \{
    "customers\_id": "df1",
    "customers\_lastname": "Ferguson",
    "customers\_firstname": "Donald",
    "customers\_email": "dff9@columbia.edu",
    "customers\_status": "ACTIVE",
    "customers\_password": "lakd;lkasddjkl;sADJ"
  \}
]

    \end{Verbatim}

    \textbf{Questions?}

\begin{figure}
\centering
\includegraphics{attachment:image.png}
\caption{image.png}
\end{figure}

    \subsubsection{Unpacking the Code Exposes some Critical
Concepts}\label{unpacking-the-code-exposes-some-critical-concepts}

\paragraph{Database Connection}\label{database-connection}

\begin{longtable}[]{@{}c@{}}
\toprule
\includegraphics{attachment:image.png}\tabularnewline
\midrule
\endhead
\textbf{Database Connection}\tabularnewline
\bottomrule
\end{longtable}

"In computer science, a database connection is the means by which a
database server and its client software communicate with each other. The
term is used whether or not the client and the server are on different
machines." (https://en.wikipedia.org/wiki/Database\_connection)

\begin{itemize}
\tightlist
\item
  Database systems and products provide client libraries and SDKs that
  simplify using the database APIs.
\end{itemize}

\begin{verbatim}
let config = {
    db_connect_info: {
        host: "localhost",
        user: "dbuser",
        password: "dbuser"
    },
    data_table: "e6156.customers",
    primary_key: "customers_id"
};
\end{verbatim}

In this simple example, the connection information is: - Host (Internet
location of the database server)

\begin{itemize}
\item
  User and password to associate with requests on this connection
  (authentication, authorization)
\item
  There are many other "command and control" properties associated with
  a connection.
\end{itemize}

    \begin{Verbatim}[commandchars=\\\{\}]
{\color{incolor}In [{\color{incolor}2}]:} \PY{n+nx}{d}\PY{p}{.}\PY{n+nx}{get\PYZus{}db\PYZus{}connection}\PY{p}{(}\PY{p}{)}
\end{Verbatim}


    \begin{Verbatim}[commandchars=\\\{\}]
db\_connect\_info =  \{
  "host": "localhost",
  "user": "dbuser",
  "password": "dbuser"
\}
Created DB connection

    \end{Verbatim}

\begin{Verbatim}[commandchars=\\\{\}]
{\color{outcolor}Out[{\color{outcolor}2}]:} Connection \{
          domain: null,
          \_events: \{\},
          \_eventsCount: 0,
          \_maxListeners: undefined,
          config: 
           ConnectionConfig \{
             host: 'localhost',
             port: 3306,
             localAddress: undefined,
             socketPath: undefined,
             user: 'dbuser',
             password: 'dbuser',
             database: undefined,
             connectTimeout: 10000,
             insecureAuth: false,
             supportBigNumbers: false,
             bigNumberStrings: false,
             dateStrings: false,
             debug: undefined,
             trace: true,
             stringifyObjects: false,
             timezone: 'local',
             flags: '',
             queryFormat: undefined,
             pool: undefined,
             ssl: false,
             multipleStatements: false,
             typeCast: true,
             maxPacketSize: 0,
             charsetNumber: 33,
             clientFlags: 455631 \},
          \_socket: undefined,
          \_protocol: 
           Protocol \{
             domain: null,
             \_events: \{\},
             \_eventsCount: 0,
             \_maxListeners: undefined,
             readable: true,
             writable: true,
             \_config: 
              ConnectionConfig \{
                host: 'localhost',
                port: 3306,
                localAddress: undefined,
                socketPath: undefined,
                user: 'dbuser',
                password: 'dbuser',
                database: undefined,
                connectTimeout: 10000,
                insecureAuth: false,
                supportBigNumbers: false,
                bigNumberStrings: false,
                dateStrings: false,
                debug: undefined,
                trace: true,
                stringifyObjects: false,
                timezone: 'local',
                flags: '',
                queryFormat: undefined,
                pool: undefined,
                ssl: false,
                multipleStatements: false,
                typeCast: true,
                maxPacketSize: 0,
                charsetNumber: 33,
                clientFlags: 455631 \},
             \_connection: [Circular],
             \_callback: null,
             \_fatalError: null,
             \_quitSequence: null,
             \_handshakeSequence: null,
             \_handshaked: false,
             \_ended: false,
             \_destroyed: false,
             \_queue: [],
             \_handshakeInitializationPacket: null,
             \_parser: 
              Parser \{
                \_supportBigNumbers: false,
                \_buffer: <Buffer >,
                \_longPacketBuffers: [],
                \_offset: 0,
                \_packetEnd: null,
                \_packetHeader: null,
                \_packetOffset: null,
                \_onError: [Function: bound handleParserError],
                \_onPacket: [Function: bound ],
                \_nextPacketNumber: 0,
                \_encoding: 'utf-8',
                \_paused: false \} \},
          \_connectCalled: false,
          state: 'disconnected',
          threadId: null \}
\end{Verbatim}
            
     Changing the connection: - I made a similar database on
\href{https://aws.amazon.com/rds/}{Amazon Relational Data Service}

\begin{itemize}
\tightlist
\item
  Created a table.
\end{itemize}

    \begin{figure}
\centering
\includegraphics{attachment:image.png}
\caption{image.png}
\end{figure}

    \begin{Verbatim}[commandchars=\\\{\}]
{\color{incolor}In [{\color{incolor}8}]:} \PY{n+nx}{config4} \PY{o}{=} \PY{p}{\PYZob{}}
            \PY{n+nx}{db\PYZus{}connect\PYZus{}info}\PY{o}{:} \PY{p}{\PYZob{}}
                \PY{n+nx}{host}\PY{o}{:} \PY{l+s+s2}{\PYZdq{}columbiae6156.cqgsme1nmjms.us\PYZhy{}east\PYZhy{}1.rds.amazonaws.com\PYZdq{}}\PY{p}{,}
                \PY{n+nx}{user}\PY{o}{:} \PY{l+s+s2}{\PYZdq{}dbuser2\PYZdq{}}\PY{p}{,}
                \PY{n+nx}{password}\PY{o}{:} \PY{l+s+s2}{\PYZdq{}dbuser2\PYZdq{}}
            \PY{p}{\PYZcb{}}\PY{p}{,}
            \PY{n+nx}{data\PYZus{}table}\PY{o}{:} \PY{l+s+s2}{\PYZdq{}cloude6156.customers\PYZdq{}}\PY{p}{,}
            \PY{n+nx}{primary\PYZus{}key}\PY{o}{:} \PY{l+s+s2}{\PYZdq{}customers\PYZus{}id\PYZdq{}}
        \PY{p}{\PYZcb{}}\PY{p}{;}
        
        \PY{n+nx}{d4} \PY{o}{=} \PY{k}{new} \PY{n+nx}{Dao}\PY{p}{(}\PY{n+nx}{config4}\PY{p}{)}\PY{p}{;}
        \PY{n+nx}{d4}\PY{p}{.}\PY{n+nx}{get\PYZus{}by\PYZus{}id}\PY{p}{(}\PY{l+s+s1}{\PYZsq{}dv1\PYZsq{}}\PY{p}{)}\PY{p}{;}
\end{Verbatim}


    \begin{Verbatim}[commandchars=\\\{\}]
execute\_query (enter), statement = SELECT * FROM cloude6156.customers where customers\_id = ? 
values =  [
  "dv1"
]
db.execute\_query: After calling con create.
Connected to DB
db.execute\_query: After query, result = [
  \{
    "customers\_id": "dv1",
    "customers\_lastname": "Vader",
    "customers\_firstname": "Darth",
    "customers\_email": "dv@deathstar.empire.gov",
    "customers\_status": "ACTIVE",
    "customers\_password": ""
  \}
]

    \end{Verbatim}

\begin{Verbatim}[commandchars=\\\{\}]
{\color{outcolor}Out[{\color{outcolor}8}]:} [ RowDataPacket \{
            customers\_id: 'dv1',
            customers\_lastname: 'Vader',
            customers\_firstname: 'Darth',
            customers\_email: 'dv@deathstar.empire.gov',
            customers\_status: 'ACTIVE',
            customers\_password: '' \} ]
\end{Verbatim}
            

    % Add a bibliography block to the postdoc
    
    
    
    \end{document}
